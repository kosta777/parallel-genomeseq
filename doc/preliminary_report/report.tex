% IEEE standard conference template; to be used with:
%   spconf.sty  - LaTeX style file, and
%   IEEEbib.bst - IEEE bibliography style file.
% --------------------------------------------------------------------------

\documentclass[letterpaper]{article}
\usepackage{spconf,amsmath,amssymb,graphicx}
\usepackage{array}

% Example definitions.
% --------------------
% nice symbols for real and complex numbers
\newcommand{\R}[0]{\mathbb{R}}
\newcommand{\C}[0]{\mathbb{C}}

% bold paragraph titles
\newcommand{\mypar}[1]{{\bf #1.}}

% Title.
% ------
\title{Parallelizing the Smith-Waterman algorithm for genome sequencing}
%
% Single address.
% ---------------
\name{Kosta Stojiljkovic, Han Yao Choong, Langwen Huang, Benjamin Langer} 
\address{Progress report - Design of Parallel and High-Performance Computing - Group 15}

% For example:
% ------------
%\address{School\\
%		 Department\\
%		 Address}
%
% Two addresses (uncomment and modify for two-address case).
% ----------------------------------------------------------
%\twoauthors
%  {A. Author-one, B. Author-two\sthanks{Thanks to XYZ agency for funding.}}
%		 {School A-B\\
%		 Department A-B\\
%		 Address A-B}
%  {C. Author-three, D. Author-four\sthanks{The fourth author performed the work
%		 while at ...}}
%		 {School C-D\\
%		 Department C-D\\
%		 Address C-D}
%

\begin{document}
%\ninept
%
\maketitle
%

\section{Introduction}

Genome sequencing has become a promising field tool that is hypothesized to facilitate researchers to find the root cause (and subsequently the cure) for a number of genetic diseases in the future. \cite{Maekinen.2015}

% Diviation from proposal
The investigation into three different algorithms in the initial proposal (see attached below) was discarded in favor of a more thorough and deeper look into one of them (namely Smith-Waterman). From a computer science perspective all of these algorithms are rather similar, consequently the team decided to focus its efforts on the one which is most relevant in the given domain.

\section{Parallelization strategy}
1) Fine-grained approach 

2) Coarse-grained approach


\section{Work split}



\begin{center}
	\begin{tabular}{ | m{1.5cm} | m{6cm} | } 
		\hline
		Name & Main focus* \\ 
		\hline
		Kosta & Coarse-grain approach, MPI \\ 
		Langwen & CMake setup, Matrix transformation \\ 
		Han & Bioinf input, OpenMP \\
		Benjamin & Testing, base algorithm implementation \\
		\hline
	\end{tabular}
\end{center}
*Various teammates also contributed actively to other fields.



% References should be produced using the bibtex program from suitable
% BiBTeX files (here: bibl_conf). The IEEEbib.bst bibliography
% style file from IEEE produces unsorted bibliography list.
% -------------------------------------------------------------------------
\bibliographystyle{IEEEbib}
\bibliography{bibl_conf}

\end{document}

